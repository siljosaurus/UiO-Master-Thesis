
The thesis has discussed how one can design meaningful interactive experiences in a museum space by analyzing a selection of interactive installations and exhibition spaces in local museums. In addition, the thesis has considered the development and shift in museology practice against ubiquitous computing and design as meaning-making in museums, aimed to forge a stronger connection between the HCI/ IxD field and museum studies. This has been conceptually explored by using sustainability and the Anthropocene as a discourse representative of a contemporary ongoing public debate or topic that museums being institutions of knowledge, have incentive and means to engage in by communicating the science of climate change. 

We presented the making of a theoretical framework, and then we applied the framework in an analysis session of 21 interactive installations. From this analysis we obtained knowledge that helped revise the framework, and we derived a list of patterns that through a description of dialogical visitor behaviour objectified meaningfulness.