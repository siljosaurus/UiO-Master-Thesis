
This thesis has discussed how one can design meaningful interactive experiences in a museum space that addresses sustainability by analyzing a selection of interactive installations and exhibition spaces in local museums. In addition, the thesis has considered the development and shift in museology practice against ubiquitous computing and design as meaning-making in museums, aimed to forge a stronger connection between the HCI/ IxD field and museum studies. This has been conceptually explored by using sustainability and the Anthropocene as a discourse representative of a contemporary ongoing public debate or topic that museums being institutions of knowledge, have incentive and means to engage in by communicating the science of climate change. 

The first contribution is a framework that can be used to identify and analyse dialogic relations between visitor and interactive installations. The second contribution is a list of 21 dialogical installation patterns that function to objectify meaningfulness as a quality that you can design for.

