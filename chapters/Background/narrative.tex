\section{Telling, showing, showing off}
\section{The value factory}
\section{The talking museum}

\subsection{Discursivity and cultural moralism}
Discursivity, most notably rhetoric imbricated with narrative is in effect a crucial aspect of the museum institution \autocite[p. 205]{Thi_book}, it is the core of the idea of exhibition.

The museum is an attractive object of study, because it requires interdisciplinary analysis, it has the debate on aesthetics at its core, and that it is essentially a social institution (Thi, p. 202) \autocite[p. 202]{Miekebal_book}. Mieke Bal account for and describe the issue of cultural imperialism in museums, exemplifying case studies related to natural history types of museums that conserves and display ethnic objects and artefacts representing cultures and cultural properties from the past. The ethnographic museum is clearly the most obviously politically charged institution, and it poses the immediate problem of cultural property and collective ownership (Thi, p. 202). It raises the question if former colonists are entitled to hold onto objects taken by their ancestors from former colonies, or should they give these back to the country of origin the ancestors of whose inhabitants were their original owners?

I think this (as a sort of ethical) perspective is necessary to have in this context, because the intention of making a meaningful interactive experience is to strengthen the message conveyed by the museum. Both the designer and especially the museum need to be aware of and able to answer moral and ethical questions in terms of what the message they convey actually conveys. And that the act of strengthening that impression in a greater way sends people out of the museum with reflections and thoughts that leads to social action (climate action, consciousness).
