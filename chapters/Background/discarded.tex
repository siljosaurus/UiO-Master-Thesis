Our lives and our societies are filled with the past. It is used and abused politically, to defend and condemn actions, to argue for right and wrong and to explain why the world and society are as they are. Therefore, it is important that we have systematic knowledge of the past - and that we can take part in critical discussions about how history is produced and used.

What can we really know about what has happened before? How do we arrive at this knowledge? Why is it constantly necessary to rewrite history? As a history student, these are some of the questions you should consider. You will learn to be critical of what we "know" about the past and acquire knowledge that forms a broad basis for understanding the person and the contexts in which he has lived and lives.




Surveys from, for example, the USA show that museums score higher on trust than other institutions, such as media and government agencies, and that credibility of museums continues to increase during the pandemic \autocite{impact_2020}. As important actors in the public discourse, it is crucial for museums in the future that they defend this position and further develop it. It is essential to secure a position where a room is created for different opinions to meet and break and counter the emergence of echo chambers, conspiracy theories, and false news \autocite[p. 41]{melding23}.


How to better “read”, describe and understand a museum, exhibition or a specific installation? These writings have formatively gudied datagathering efforts. \autocite{Miekebal_book} writes about cultural analysis and discourses in the museum, 