\section{Museum's role in society}
In the recommendation letter report \emph{“Museums in the society - trust, things and time,”} the standing committee of Cultural Affairs presents the overall political direction for Norwegian museum policy towards the year 2050 (Figure 1.1). The report establishes that Norwegian museum institutions take aim to express both historical and current developments in society. And that museum institutions play an important role in our own time’s understanding of ourselves - both who we have been, who we are, and whom we want to be \autocite[p. 7]{melding23}. Modern museum operations aim to actively turn to the general public with the objective to build and share knowledge, increase enlightenment and cultivate cultural capital. Where the early museum institutions, first and foremost, were accessible to a small number of privileged class members, it is now expected that museums purposefully reach out to be open and accessible to all residents and age groups \autocite[p. 14]{melding23}. In Norway, museums are increasingly understood as both knowledge and social institutions, where dissemination and exhibition practice are curated accordingly. At the same time, museums have emerged as an alternative learning arena for school children, first regarding history teaching, but later in many other subject areas as well. In a society where polarization and public debate are intensifying, arenas that have the public's confidence in being able to nuance and disseminate different perspectives are needed \autocite[p. 7]{melding23}.

\begin{figure}[H]   
\centering
\includegraphics[width=8cm]{pictures/Introduction/stortingsmelding_hoykant.png}
\caption{The recommendation letter report Meld.St.23 \emph{“Museums in the society - trust, things and time"}} {\autocite[p. 1]{melding23}}
\end{figure}

There is a saying among historians that \emph{"the past teaches us about the present"}. History is a subject that is extra rich in perspectives, explanations, and ideas about how people have lived, thought, and acted. It positions us to see patterns that might otherwise be invisible in the present – thus providing a crucial perspective for understanding and solving current and future problems \autocite{UW_website}. Being a knowledge institution, museums have the power to both define and showcase relevant historical events as a perspective to inform present societal issues and debates.

\section{Research question and framing of project}
The aim of this study is to gain insight to try answer how one can design interactive meaningful experiences in a museum space. To understand this, several installations have been analysed as a way to objectify \emph{meaningfulness} as a quality that you can design for. In the HCI community, meaningfulness is recognised as a quality in made products or artefacts that not only make people more efficient and effective, but through use in activities, relationships, routines, and rituals, become meaningful \autocite{zimmerman_designing_2009}. This thesis fills a gap in the literature where the term is explored through a museum context for designers working with exhibition- and experience-design, where meaningfulness is understood as a dialogical design quality that promote dialogue. Special attention has been given to museums that disseminate knowledge on sustainability, aiming to encourage social action, like climate consciousness or climate action as examples of meaningful behaviour museums addressing contemporary discourses want to encourage. The thesis project is framed through the research question:

\begin{quote}
    How can one design meaningful interactive experiences in a museum space?
\end{quote}

In the attempt to answer how one can design meaningful interactive experiences in a museum space, I have composed a theoretical framework that frames a place-centred understanding of meaningfulness. The framework can be used in exhibition spaces, where one through analysis of installations can identify how one can support or extend the exhibition to become more dialogic and thus, meaningful. An interactive installation or experience have the potential to reinforce the message conveyed in a manner that give rise to thought-provoking or significant reflections that last long after the museum visit. The hypothesis is that there lies value for the museum to judge whether or not their exhibitions promote dialogue in compliance with the museums agenda and vision.

\section{Chapter overview}
The thesis is divided into three parts; Introduction, Design Process, and Review/ Evaluation. The report structure reflects the methodological nature of doing Research through Design, where the Design Process is evidence of the practical activities shaping the thesis project. If not referenced otherwise, all pictures in this thesis come from a shared photo library the research buddies and I have built up and accumulated during fieldwork. The same goes for illustrations. If not referenced else-wise, they are illustrated by me.

\subsubsection{Chapter 2: Modern museums and sustainability}
Chapter 2 is a focused selection of terminology and concepts attained from the literature review that has been conducted throughout/during the thesis project. It is structured in the hope that designers interested in the topic can adopt the concepts as a vocabulary as a mean of understanding some of what goes on in the museum world. The literature review has nonetheless been crucial for the thesis evolution. On the one hand, it has influenced the research framing and interest, shaped and refined the research question, and leveraged the vocabulary when documenting/ recounting an exhibition- and installation experience. On the other hand - influencing the convergent and divergent thought processes when going back-and-forth of the practical and theoretical work, the designer- and the researcher role.

\subsubsection{Chapter 3: Three approaches to place-centred design }
Chapter 3 is a continuation of the literature review. Where Chapter 2 indirectly affect the thesis evolution, Chapter 3 present three theoretical approaches that have formatively guided e.g. data-gathering guides during fieldwork, both for observations, interviews and analytical critique. The chapter presents three theoretical approaches that frames a place-centred understanding of meaningfulness, which is accounted for in the next chapter.

\subsubsection{Chapter 4: A new way of designing for meaningfulness?}
In Chapter 4, we build upon and borrow concepts from the three theoretical approaches presented in Chapter 3, and presents the philosophical background on the framework as a proposal of a new way of designing for meaningfulness.

\subsubsection{Chapter 5: Methodology}
In Chapter 5 you can read about the methodological approach to Research through Design adopted through this thesis. 

\subsubsection{Chapter 6: Design Process}
This chapter account for the design activities and exhibitions that have been visited along with the research buddies.

\subsubsection{Chapter 7: Analysis}
In this chapter we will go through how the installations have been analysed and present the findings. Then we will revisit the theoretical framework and reflect on the generated knowledge from the practical application of the framework.


\subsubsection{Chapter 8: Discussion}
In this chapter we look back at the thesis project and discuss how one can design for meaningful interactive experiences in a museum space.


\subsubsection{Chapter 9: Conclusion}
Concluding the thesis report.