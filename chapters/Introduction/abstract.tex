This thesis reports from an analysis of 21 interactive installations, seeking to answer how one can design meaningful interactive experiences in a museum space that addresses sustainability. We present several patterns indicating how interactive installations engage visitors with their surroundings or fellow visitors in a museum space. The patterns stem from the application of a theoretical framework developed throughout this thesis project that designers can use to identify and analyze dialogic relations between visitors and installation in a museum space.

The thesis aims to build a stronger connection between the HCI/ IxD field and museum studies by presenting literature connecting sustainability and the Anthropocene as a discourse representative of a contemporary ongoing public debate against museum experience design. Considering the development and shift in museology practice, museums being institutions of knowledge, have both the means and incentives to position themselves to be a place where visitors can meaningfully engage with the science of climate change. % Og være forgjengere på hvordan interaktivitet, teknologi kan brukes på meningsfulle måter.

- Mer utfyllende om problemstillingen.! og hva jeg har funnet. bygge ut mer fra chapter . hva tar denne artikkelen for seg.
konklusjon = sy sammen 