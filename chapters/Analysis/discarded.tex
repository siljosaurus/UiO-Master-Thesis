The main purpose of this analysis-oriented part is to contribute with knowledge that make visible dialogic qualities between interactive installations and visitor-experience.

The practical application of this analytical tool is something like this (the tables underneath), where I have plotted in the different installations and exhibitions to their respective tables. A thorough walk-through of the analysis is accounted for in Chapter 9: Analysis.


\section{notes}

\par Then again, we can look at what dialogic qualities the installations turns out to have little "relationalness", which is a dialogic principle/ quality that involves the Docents in the museum for example, or relational qualitites.


heat map: Categorising it this way opened up for looking at the data-set in correlation with each theory separately, making it possible to see overall trends in the data-set according to the theories. 

Figure 9.1 illustrates how each theory contribute to the understanding of meaningfulness, as explained in chapter 6. The lines represents what we strive to find during this analysis; dialogic relations between visitor and installation, and objectifying meaningfulness as a quality that you can design for.



\textbf{Hybrid place table}
Seeing that all written experience accounts are based on the hybrid place dimensions, this table was quite straight-forward to work with. What we see from this table is 

Wanting to get a place-centred understanding of the installations, one whole side/edge of the triangle-framework I present are built upon the four dimensions presented by \autocite{hybridplace_ciolfi}; the \emph{personal}, \emph{cultural}, \emph{physical/structural}, and \emph{social} dimension, accounted for in chapter 3.


\textbf{place as dialogue qualities}

Wanting to get a better understanding of the visitor experience with attention to the sensory transactions and the way the experience transform in the telling (what is the things you remember), I utilise \autocite{spaceplace_ciolfi} "dialogical ontology", or as I call it, the list of place-related qualities. 

\textbf{Dissemination}
Then, the sense-making strategies make up



\begin{table}[h]
\centering
\begin{tabular}{l | l| l}
\textbf{Type} & \textbf{Name} & \textbf{Museum}\\
\hline
Literature review & Museums and Sustainability & \\
Prototyping & The relationship between human and nature \\
Presentation & Exploring input through plants & Klimahuset\\
Exhibition & Liquid Life & Kistefoss \\
Observation & Climate Dialogue w/ 2 school classes & Klimahuset\\
Interview & Concept Developer & MUNCH\\
\end{tabular}
\caption{Other fieldwork}
\label{tab:abc}
\end{table}


Pattern from Yuko Mohri:


- Visitors influence eachothers visits. They "acquire" the way we are, and they make us see things we did not see at first. Munch concept developer is also talking about this!


What I have learned by looking at the radar charts so far is how the different theories fulfill, or complement eachother. The way I have gone forward in looking at the radar charts is as following:
\par I'll start by looking at the hybrid place radar chart, noticing how the personal and physical dimension is fulfilled, while the social and cultural dimension is very little fulfilled. What does this mean? According to the Norwegian museum policy strategic thinking, it is wanted that museums transition from the personal dimension to the more social and cultural dimension. The fact that installations in my analysis shows presence in the physical dimension is positive, in terms of enabling the personal dimension, experience-wise, to involve more tangible or at least dynamic experiences in the museum space.
\par Then, if we shift focus for a second to look into the sense-making radar chart, we see that one of the corners that is fulfilled by almost all installations in this analysis - support for sharing is fulfilled. How come, that even though the social dimension is not fulfilled while almost all installations, in terms of sense-making have good support for sharing? 





\begin{table}[h]
\centering
\begin{tabular}{| p{1cm} | p{11cm}| }
\hline
\textbf{Nr} & \textbf{Pattern} \\
& \textit{Place as dialogue }\\
\hline
6 & in the centre of a variety of sense-making practises \\
\hline

\end{tabular}
\caption{Findings table}
\label{tab:abc}
\end{table}

