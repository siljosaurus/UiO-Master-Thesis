% "meaningfulness" up against the field's definition of meaningfulness. 


\subsection{Use of patterns}
"My use of design patterns deviates fairly widely from their intended purpose of documenting and making explicit design conventions. The design patterns from the individual products are not particularly valuable to other designers, because they do not document a general solution. However, they worked very well to illuminate the link between design intention (application of product design theory) and the designed artifact, allowing the similarities in the application of theory across the very different design projects to be seen." \autocite[p. 402]{zimmerman_designing_2009}.

The primary purpose of this thesis has been to contribute with knowledge that makes visible dialogic relations between interactive installations and visitor-experience in the museum space, to attempt to answer how one can design meaningful interactive experiences in a museum space. "By making different things intended to address the same problematic situation, RtD can reveal design patterns \autocite{Alexander_book} around problem framings, around specific interactions, and around how theory can be operationalized" \autocite[p. 178]{zimmerman_research_2014}.


"Thanks to the development of ubiquitous and pervasive technologies, research focusing on the design of novel interactive artefacts has recently become more concerned with studying and understanding the spatial properties of the world" \autocite[p. 159]{hybridplace_ciolfi}


One of the things which in my opinion make research through design stand out, is how the researcher's subjective but theoretically grounded opinions are validated as much as the user. Take writing as an example- writing is a creative process. The researcher are a good writer specialised in the translation from academic readings to . A lot of knowledge is produced during the writing process, and in traditional science this writing isnt appreciated enough. Writing is a creative process (method) where a lot of knowledge is processed and produced. In my opinion, RtD is recognising and acknowledging, legitimising, this knowledge producing process. Research, of all diciplines, should aim to rightfully showcase the knowledge-process that happens through writing and reading..


\subsection{What designers make of what they see}
% If relevant, add citation to bibliography
Jane F. Suri writes in her article Poetic Observation: What Designers Make of What They See(2011), that “Designers need to interpret what they see (and otherwise sense) in ways that will lead to design outcomes.” Which is especially true for me and the outcome of my thesis, and also why I need to be aware and critical of my role and values as a researcher, because  I am the one shaping and deciding how and what the aspects of the installation that adresses sustainability should be. Suri further emphasises how interpreting what we see is both a personal and a social process, and that “In our own individual ways, all of us pay attention to what’s meaningful and interesting to us. While we might guess that certain activities will be fruitful, not everyone sees the same things or finds them equally useful.”(Suri, 2011, p. 18).

Jane Fulton-Suri, reflecting on her training in social science followed by her experience working at a design consultancy, identified a gap between design with its focus on te and social science research with its focus on the past and present \autocite[p. 167]{zimmerman_research_2014}.he futur



\subsection{form, fit, context: learning about the context }
I believe, as a first step, it is important to try to be as fully immersed in the existing context of ‘what Klimahuset is’, as possible. Before thinking about what form the installation should have, I need a thorough understanding of the context the installation is to be placed in, to be able to start shaping the form. The existing context will also set some constraints for what kind of fit the installation will have in the context, a fit that will be continually evaluated throughout the design process. 

In the documents I have gotten access to from Klimahuset, there is a great focus on the buildings architecture. Design and architecture goes hand in hand, and the architectural outline of the exhibition space, sets creative constraints and limits for what the installation can be. It is also the “opportunity space” (mulighetsrom) I have available in the design of the installation. It is therefore important for me to know the “material”, thoughts and reasoning behind it.


\subsubsection{Has research become a status issue?}
Fraylings questioning of research becoming a status issue have been a perspective in the back of my mind 

\emph{"Where the tradition is concerned, one interesting question is why people want to call it research with a big 'r' at all. Whats the motivation? True, research has become a political or resource issue, as much as an academic one. And, as a slight digression, it always amuses me to see the word 'academic' used as a pejorative - by people who themselves earn their livings within the academy. Research has become a status issue as much as a conceptual or even practical one. And that - I must confess - worries me. There may well be opportunities for research within the expressive tradition, but they need dispassionate research, rather than heated discussion about status, class and reverse snobbery"} \autocite[p. 5]{frayling_1994}.




Research through design is a "radical" way of doing science. I think it is groundbreaking and paving a more authentic and selfless way of doing science. One of the things which makes RtD a different methodology than other is the focus on the designer/ scientists knowledge, and how subjective but theoretically grounded opinions and forces fulfil and complement the science project. 

Writing is a creative process, and scientist/ researchers are good writers. A lot of knowledge is produced during the writing process, and in traditional science this writing isnt appreciated enough. Writing is a creative process (method) where a lot of knowledge is processed and produced. In my opinion, RtD is recognising and acknowledging, legitimising, this knowledge producing process. Research, of all diciplines, should aim to rightfully showcase the knowledge-process that happens through writing and reading..  (literature review..)

Zimmerman and Forlizzi are one of "the big names" in terms of research through design in interaction design, hci, and I know they exist. However, I find their approach to be very analytical, and actually, a bit difficult to apply in a real-time project. Which is why I'd rather use a more open-ended way of describing my thesis process.


\section{Sustainability}

Limitation?, scientist role on the debate and as a researcher, specifically related to the climate debate:
“What’s the motivation? True, research has become a political or resource issue, as much as an academic one. And, as a slight digression, it always amuses me to see the word ‘academic’ used as a pejorative - by people who themselves earn their livings within the academy. Research has become a status issue, as much as a conceptual or even practical one. And that - I must confess - worries me.” \autocite[p. 5]{frayling_1994}. This is quite interesting, and in some way, it could be something we see the consequences of in regards to the ongoing climate debates where scientists are doubted and not trusted. That the science they predict feels more like a hoax. Perhaps this mistrust is somewhat manifested through this stereotypical description that Frayling describes.