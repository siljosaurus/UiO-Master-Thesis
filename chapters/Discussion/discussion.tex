In this chapter we look back at the thesis project and discuss it.

\begin{itemize}
    \item Hvordan svarer jeg på forskningsspørsmålet mitt og hvordan bidrar bidraget til fagfeltet?
    \item svare på forskningsspørsmålet opp mot litteraturen,
    \item diskutere findings opp mot litteraturen, 
    \item Limitations, 
\end{itemize}


\section{The nature of the contribution}



\subsection{Research as a creative, generative process}

\textbf{Research as a creative, generative process}
Documentation --> the researchers basis for contributions (\emph{the research}). In research through design, the documentation also play a creative and generative role - because it is involved as part as a design process just as much as the research process.
In a way, I see the research question as something "levende", fluid, that is not static. 
Forskning er en generativ prosess - science is a generative and creative process.
Perhaps also say something about autoethnography/ biography?
"The inner-organisation of the portfolio is annotated to make it communicable to other people (researchers, designers, users, whoever). Effective annotation will also enable comparative work" \autocite[p. 72]{bowers_annotated_2012}. 



\subsection{Research through design, methodology}

The methodological approach in this thesis reflects the same challenges that is discussed in the literature in terms of doing 

The methodological approach in this thesis reflects the same challenges that is discussed in the literature in terms of doing research through design. Not only is there little 

A researcher's methodology allows the reader to understand the approach and methods used to reach conclusions. This is important in any academic discipline, an

A research methodology gives research legitimacy and provides scientifically sound findings. It also provides a detailed plan that helps to keep researchers on track, making the process smooth, effective and manageable. A researcher's methodology allows the reader to understand the approach and methods used to reach conclusions.

To showcase and give reason for the process and progress of any aca

In academic papers



The methodology for this study reflects the challenges resulting from the intersection between installation design, museum facilitating attitude change and sustainability as a contemporary, and complex (!), discourse. 

Liker ikke "fase-tenkningen" dems. Its difficult to lock and close the activities in a box!!

Research through design is a "radical" way of doing science. I think it is groundbreaking and paving a more authentic and selfless way of doing science. One of the things which makes RtD a different methodology than other is the focus on the designer/ scientist@s knowledge, and how subjective but theoretically grounded opinions and forces fulfil and complement the science project. 

Writing is a creative process, and scientist/ researchers are good writers. A lot of knowledge is produced during the writing process, and in traditional science this writing isnt appreciated enough. Writing is a creative process (method) where a lot of knowledge is processed and produced. In my opinion, RtD is recognising and acknowledging, legitimising, this knowledge producing process. Research, of all diciplines, should aim to rightfully showcase the knowledge-process that happens through writing and reading..  (literature review..)

Zimmerman and Forlizzi are one of "the big names" in terms of research through design in interaction design, hci, and I know they exist. However, I find their approach to be very analytical, and actually, a bit difficult to apply in a real-time project. Which is why I'd rather use a more open-ended way of describing my thesis process. 

I find Zimmerman and Forlizzi's model (define, discover, synthesize, construct, refine, reflect) difficult to be a good fit in terms of RtD's practical nature. Thinking and structuring the process in phases mean 



\section{The "practical" researcher}
Early in the book Doing Ethnographies, we are introduced to the concepts of ‘the detached researcher’ and of ‘the pure subject’ which together adresses the power relations between the researcher and the research-subject. “Research is an embodied activity that draws in our whole physical person, along with all its inescapable identities. (…) what we bring to the research affects what we get .“(Crang and Cook, 2007, p. 9). I have most definitely brought with me biases, opinions and attitudes toward the theme of sustainability, [...] which definitely framed the way I looked and inquired into the topic,

Limitations, scientist role on the debate and as a researcher, specifically related to the climate debate:
“What’s the motivation? True, research has become a political or resource issue, as much as an academic one. And, as a slight digression, it always amuses me to see the word ‘academic’ used as a pejorative - by people who themselves earn their livings within the academy. Research has become a status issue, as much as a conceptual or even practical one. And that - I must confess - worries me.” \autocite[p. 5]{frayling_1994}. This is quite interesting, and in some way, it could also be something we see the consequences of in regards to the ongoing climate debates where scientists are doubted and not trusted. That the science they predict feels more like a hoax. Perhaps this mistrust is somewhat manifested through this stereotypical description that Frayling describes.

\subsection{What designers make of what they see}
% If relevant, add citation to bibliography
Jane F. Suri writes in her article Poetic Observation: What Designers Make of What They See(2011), that “Designers need to interpret what they see (and otherwise sense) in ways that will lead to design outcomes.” Which is especially true for me and the outcome of my thesis, and also why I need to be aware and critical of my role and values as a researcher, because  I am the one shaping and deciding how and what the aspects of the installation that adresses sustainability should be. Suri further emphasises how interpreting what we see is both a personal and a social process, and that “In our own individual ways, all of us pay attention to what’s meaningful and interesting to us. While we might guess that certain activities will be fruitful, not everyone sees the same things or finds them equally useful.”(Suri, 2011, p. 18).

Jane Fulton-Suri, reflecting on her training in social science followed by her experience working at a design consultancy, identified a gap between design with its focus on the future and social science research with its focus on the past and present \autocite[p. 167]{zimmerman_research_2014}.

\subsubsection{Has research become a status issue?}

Fraylings questioning of research becoming a status issue have been a perspective in the back of my mind 

\emph{"Where the tradition is concerned, one interesting question is why people want to call it research with a big 'r' at all. Whats the motivation? True, research has become a political or resource issue, as much as an academic one. And, as a slight digression, it always amuses me to see the word 'academic' used as a pejorative - by people who themselves earn their livings within the academy. Research has become a status issue as much as a conceptual or even practical one. And that - I must confess - worries me. There may well be opportunities for research within the expressive tradition, but they need dispassionate research, rather than heated discussion about status, class and reverse snobbery"} \autocite[p. 5]{frayling_1994}.

\par

\begin{comment}
\subsection{form, fit, context: learning about the context }

I believe, as a first step, it is important to try to be as fully immersed in the existing context of ‘what Klimahuset is’, as possible. Before thinking about what form the installation should have, I need a thorough understanding of the context the installation is to be placed in, to be able to start shaping the form. The existing context will also set some constraints for what kind of fit the installation will have in the context, a fit that will be continually evaluated throughout the design process. 


In the documents I have gotten access to from Klimahuset, there is a great focus on the buildings architecture. Design and architecture goes hand in hand, and the architectural outline of the exhibition space, sets creative constraints and limits for what the installation can be. It is also the “opportunity space” (mulighetsrom) I have available in the design of the installation. It is therefore important for me to know the “material”, thoughts and reasoning behind it.
\end{comment}
