In this chapter we look back at the thesis project and discuss how one can design for meaningful interactive experiences in a museum space.

\section{The nature of the contributions}
This thesis has two research contributions, one theoretical and one practical:
\begin{itemize}
    \item Theoretical contribution: framework used to identify and analyse dialogic relations between visitor and interactive installations.
    \item Practical contribution to HCI/ IxD field: dialogical installation patterns that objectify meaningfulness as a quality that you can design for.
\end{itemize}


\section{Contributions to the field}

\subsection{The interaction designer's role in the museum}
Literature agrees on museums having become a place for education and learning, dialogue and debate \autocite{narrative_sitzia}, \autocite{hein_1998}, \autocite{hooper_1994}, \autocite{Roberts_1997}. An interest that is also shared politically, as evident in \autocite{melding23}. Surveys from, for example, the USA show that museums score higher on trust than other institutions, such as media and government agencies, and that credibility of museums continues to increase during the pandemic \autocite{impact_2020}. As important actors in the public discourse, it is crucial for museums in the future that they defend this position and further develop it. It is essential to secure a position where a room is created for different opinions to meet and break and counter the emergence of echo chambers, conspiracy theories, and false news \autocite[p. 41]{melding23}. 
I believe this thesis contributions could prove valuable for the broader HCI community interested in designing or analysing public installations in museums, as a resource for the designers working in exhibition contexts. We have shown how the theoretical contribution (the framework) can be applied to analyse a specific exhibition addressing one discourse. But also seen how the framework work across different installations and exhibitions addressing different discourses. The practical contribution on the other hand, the patterns, at this stage bear bear too much resemblance to the installations that they have been derived from, and therefore perhaps serve a more semiotic function in showcasing interactive and dialogic visitor behaviour between visitor and installation. They can serve as concrete ways that existing interactive dialogic installations function, and could be used thereafter of designers who during prototyping want inspiration or references as to how they can translate a conceptual notion into a concrete semiotic structure. So that they can judge or get an indication of the dialogic qualities the concept exploration inherit. 

Together, the practical and theoretical contribution synthesise an understanding of meaningfulness that is rooted in the museum and exhibition artefacts supporting or extending dialogic behaviour: conversations, reflections etc, between the visitor and installation. Literature concerned with interactivity and meaningful engagement with visitors: \autocite{mccarthy_place}, \autocite{ciolfi_designing_2012}, are often written in the context of heritage museums. While this thesis contribution is derived in the context of different museum fictions; art (MUNCH), science (Teknisk Museum), and natural history (Klimahuset). This thesis have given special attention to Klimahuset as a museum disseminating knowledge on the climate crisis, a discourse that differ from typical heritage museum dissemination discourses in the calling for action and debate. 


\subsection{Meaningful user experiences}
HCI is increasingly interested in considering how technology-use impacts the human experience of meaning \autocite{kaptelinin_technology_2018}, \autocite{light_design_2017}, as well as how to design for and foster meaningful user experiences \autocite{grosse-hering_slow_2013}, \autocite{Hassenzahl_Moments_2013}. Then comes the question, what does the interactive artefact or installation represent? As far as my literature search goes, there is not so much attention given to the role of the interactive artefact or installation in relation to the museum discourse, (the expository agency), in the HCI community. The community seem to be more interested on what makes the installation engaging, and how it can provide an engaging experience for the visitor and groups as evident in the works of \autocite{hornecker_learning_2006}, and \autocite{ciolfi_designing_2012}. There has been increased interest in the field of visitor studies and museum research in the details of visitor behaviour in museums, \autocite{hornecker_to-and-fro_2016}. I would argue that the application of this thesis's theoretical framework bring to life a new perspective in terms of making visible visitor behavior in museums. In this research context we have identified and being able to discuss dialogic behaviour between visitor and interactive artefacts/ installations. According to Ciolfi and McCarthy, (whose theoretical approach I have built upon), it is argued how frameworks that have been made to guide the design of interactive museum exhibitions, underplay aspects of visitor’s active sense making and interpretation. Ciolfi and McCarthy further argue that most practical and conceptual contributions from both museum studies and interaction design have fallen short of their potential to reflect on and design technologically mediated museum experiences partly because of the underdeveloped or under-articulated conceptualisations of visitor experience with which they work \autocite[p. 248]{mccarthy_place}. These concerns have been accounted for through the design of the theoretical framework, and make up the argumentation and aim as to how the framework captures, or frames, meaningfulness in the studied research context.

\subsection{Generating knowledge through analysis of design}
As a field, HCI must answer to what sorts of knowledge outcomes can come from objects in (art and) design projects; if we cant, we cannot legitimize RtD as a way of doing HCI research. Bardzell offer theoretical support for RtD by arguing that to legitimise and make use of research through design as \emph{research}, HCI researchers need to explore and clarify how RtD objects contribute to knowledge \autocite[p.2093]{bardzell_immodest_2015}. Along these lines, Bardzell et.al argue that while the \emph{intentions} of the object's designer are important and annotations are a good mechanism to articulate them, the critical reception of objects can be equally generative of RtD's knowledge impacts \autocite[p. 2093]{bardzell_immodest_2015}. Bardzell et.al. investigate RtD in its relation to the production of knowledge; specifically, \emph{how design objects are knowledge producers both for those that encounter them and those that design them} \autocite[p. 2093]{bardzell_immodest_2015}. 

Research through Design is still a methodological approach in its early years, with yet much to be discovered and established. In the beginning of this thesis project, I was unaware of the methodological and practical consequences of structuring my understanding through three different theories, while doing research through design. It has been a time-consuming effort to support this methodological choice and find support in the literature as to how this way of doing RtD has generated knowledge. The thesis project reflects the challenging "climate" of design research meeting design of installations (and exhibition journeys), and museums facilitating attitude change and sustainability as a contemporary, complex, discourse. Being a good fit to the thesis following a research through design approach. In many ways, I see Research through design as a "radical" way of doing science, breaking a lot of the norms linked to the traditional way, and Frayling's depiction of the stereotypical scientist. In research communities like HCI - concerned with researching in the scope of a domain and context which constantly evolves, where new technologies and ways of using technologies present complex ethical challenges and dilemmas emerge - I believe approaches like RtD are a good fit to address this type of research climate.

\subsection{Future work}
Due to the broadness of the research question, this thesis report has not covered or involved visitors meaning-making. There are opportunities for future work for theoreticians as well as designers. As such, all components in the thesis report are good starting points for further investigation. The design space in this research context remains open for moving in the direction of both Design Practise and Design Exploration. I am convinced that the theoretical concepts in the framework can be expanded or better articulated by further studying underlying concepts and ideas through Practise or Explorations. The patterns in this thesis project bear too much resemblance to the installations that they have been derived from, and therefore they perhaps serve a more semiotic function in showcasing interactive and dialogic visitor behaviour between visitor and installation. In addition, I would be interested in seeing the framework applied with a combination of both interactive and analogue installations. If the curator is looking at the exhibition as a whole, it would make sense to involve all parts of the journey in the analysis of the exhibition. And I believe it would provide insights on how to position interactivity in balance with "analogue" artefacts and artworks, for example. A general issue for future work would be to examine the meaning-making processes after the museum visit and understand more about what interactive aspects truly engage the visitor, making a strong impression that can last long after the museum visit. I also firmly believe there lies much knowledge to be shared in the intersection of museology studies and design, where designers can learn from museum experts and museum curators - and vice versa.

Regardless of which aspect of this thesis report others may find the most interesting, I hope it triggers some trains-of-thoughts of how designers through interactive installation design can take part in public discourses like the climate debate. Designing for meaning-making is essentially also designing for the future, being meaningful in itself, and in this research context it would give the interaction designers the opportunity to partake in the shaping of future public rooms where dialogue, reflection, education and learning is found.