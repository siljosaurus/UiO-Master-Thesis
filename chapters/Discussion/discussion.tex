In this chapter we look back at the thesis project and discuss how one can design for meaningful interactive experiences in a museum space that addresses sustainability.

\section{The nature of the contributions}
This thesis has two research contributions, one theoretical and one practical:
\begin{itemize}
    \item Theoretical contribution: framework used to identify and analyse dialogic relations between visitor and interactive installations.
    \item Practical contribution to HCI/ IxD field: dialogical installation patterns that objectify meaningfulness as a quality that you can design for.
\end{itemize}


\section{Contributions to the field}

This thesis contribution is valuable for the broader HCI community interested in designing or analysing public installations in museums. By presenting a theoretical framework used to both identify and analyse dialogic relations between visitor and interactive installations. 

HCI is increasingly interested in considering how technology-use impacts the human experience of meaning \autocite{kaptelinin_technology_2018}, \autocite{light_design_2017}, as well as how to design for and foster meaningful user experiences \autocite{grosse-hering_slow_2013}, \autocite{Hassenzahl_Moments_2013}.


Literature concerned with interactivity and meaningful engagement with visitors [ \autocite{mccarthy_place}, \autocite{ciolfi_designing_2012}], are often written in the context of heritage museums. While this thesis contribution is derived in the context of different museum fictions; art (MUNCH), science (Teknisk Museum), and natural history (Klimahuset). 


Literature agrees on museums having become a place for education and learning, dialogue and debate \autocite{narrative_sitzia}, \autocite{hein_1998}, \autocite{hooper_1994}, \autocite{Roberts_1997}. An interest that is also shared politically, as evident in \autocite{melding23}. 

Surveys from, for example, the USA show that museums score higher on trust than other institutions, such as media and government agencies, and that credibility of museums continues to increase during the pandemic \autocite{impact_2020}. As important actors in the public discourse, it is crucial for museums in the future that they defend this position and further develop it. It is essential to secure a position where a room is created for different opinions to meet and break and counter the emergence of echo chambers, conspiracy theories, and false news \autocite[p. 41]{melding23}.


Then comes the question, what does the interactive artefact or installation represent? As far as my literature search goes, there is not so much attention given to the role of the interactive artefact or installation in the HCI community. The community seem to be more interested on what makes the installation engaging, and how it can provide an engaging experience for the visitor and groups as evident in the works of \autocite{hornecker_learning_2006}, and \autocite{ciolfi_designing_2012}.

"For the last several years the HCI community has been undergoing a broadening of scope from usability to user experience; making things that improve the quality of people’s lives across a range of contexts. Our community has recognized that in addition to making products that make people more efficient and effective, we need to learn to make things that affect people in a variety of ways." \autocite[p. 395]{zimmerman_designing_2009}. 

\textbf{USE OF PATTERNS:} "My use of design patterns deviates fairly widely from their intended purpose of documenting and making explicit design conventions. The design patterns from the individual products are not particularly valuable to other designers, because they do not document a general solution. However, they worked very well to illuminate the link between design intention (application of product design theory) and the designed artifact, allowing the similarities in the application of theory across the very different design projects to be seen." \autocite[p. 402]{zimmerman_designing_2009}




\section{Disposition}

"Overall this paper contributes to HCI research on the technological augmentation of heritage sites both by addressing a seldom-studied domain, (...) This contribution is also of value for the broader HCI community interested in public installations, by showing how a thoughtful assembly of different interactional elements (from mobile applications, to standalone pieces and low-tech tangible components) can engage visitors and engender the establishment of personal connections." \autocite{ciolfi_designing_2012}


\begin{itemize}
    \item Contribution 1: by putting three theoretical approaches together we have made a framework (chapter 6) that proposes a new way of designing for meaningfulness. The new framework have been used to guide datagathering (chapter 8), and through analysis we have proven how we can derive a list of patterns. The framework have therefore proved useful as a tool for designers following a whole iteration in a design process.
    
    
    \item Contribution 2: the list of patterns. They can be used (by designers) in early conceptual stages as inspiration or starting points for dialogic installation designs. 
    \item These contributions is the answer to; \emph{How one can design for meaningful interactive experiences in a museum space.}
    \item
    
     \item These contributions + literature review (chapter 2), is the answer to; \emph{How can one (...) in a museum space that addresses \textbf{sustainability}}. Sustainability represents a contemporary discourse that demand active discussion and engagement. Dialogic installations are my proposal of a way that museums can address contemporary issues without being cultural moralists, simply facilitating for reflection, and dialogue in the museum, as the  in the time after the visit. 
    \item Discuss dialogic installation design up against the field's current "state" or contributions of something like that.

\end{itemize}


\section{Research as a creative, generative process (or something like this?)}
This study reflects the challenging "climate" where design research meet installation design, museums facilitating attitude change and sustainability as a contemporary, and complex (!), discourse.

\begin{itemize}
    \item Contributions to the field (HCI, interaction design, museum studies). HCI/ I.D. --> RtD as a generative and creative process. Museum studies: an attempt to objectify interactive meaningful qualities in museum spaces
    \item Discuss my term "meaningfulness" up against the field's definition of meaningfulness. 
    
    \item Documentation --> the researchers basis for contributions (\emph{the research}). In this study, the documentation have played a creative and generative role - just as much as being sound (valid) for scientific purposes like analysis. 
    \item RQ ---> in this study used as a "creative driver" for the design process. Seen as something "alive", fluid, that drives conceptual exploration just as much as the scientific exploration/ inquiry.
    \item Science is a generative and creative process. This study aim to strengthen the notion that design has a place in research, agreeing with Frayling and RtD literature saying that science + design = true. 

\end{itemize}


