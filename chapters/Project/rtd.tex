\section{Research through Design}

When it comes to the structural nature of the coming-together of this thesis, the methodology, I have followed an approach to research practise called Research through Design (RtD).

This research approach was first coined by \autocite{frayling_1994}, in a highly influential paper where he addressed the debate and confusion at the time around what research \emph{is}, what it \emph{involves}, and what it \emph{delivers}. Frayling critiques the stereotypical perceived difference between the Research field and the Art and Design field - whereas 'researching' stereotypically is seen as a cognitive practise, while art and design is seen as an expressive practise \autocite[p. 5]{frayling_1994}. Concluding that since many of the motivations and practises of the two fields are alike, there is a more productive distinction of the relations between research, art and design \autocite[p. 5]{frayling_1994}, namely: research \emph{into} art and design, research \emph{through} art and design, and research \emph{for} art and design. 

According to Frayling, Research \emph{into} art and design is concerned with historical research, aesthetic and perceptual research, and research into theoretical perspectives on art and design \autocite[p. 5]{frayling_1994}. Research \emph{for} art and design on the other hand is research "where the end product is an artefact - where the thinking is, so to speak, \emph{embodied in the artefact}, where the goal is not communicable knowledge in the sense of verbal communication, but in the sense of visual or iconic or imagistic communication"\autocite[p. 5]{frayling_1994}. Lastly, Research \emph{through} art and design is concerned with either/ or materials research, development work (i.e. customising a piece of technology to do something no-one had considered before, and communicate the results), as well as action research (i.e. where a research diary tells, in a step-by-step way, of the experiment and result), underlining how "both the diary and the report are there to \emph{communicate the results}, which is what separates \emph{research} from the gathering of reference materials" \autocite[p. 5]{frayling_1994}.
\par
\textbf{The thing is, during this thesis project I have only done one 'designing' activity (ref. section 8.3: Exploring input through plants), that would suggest a fit into Frayling's description of research through design, as a type of development work. As I hope will become clear, the main reason for this thesis to be placed into the RtD approach by Frayling, is because the process diary and thesis report as a whole documents the subjective, but theoretically grounded progression I (as a researcher) have gone through in terms of gaining knowledge and understanding as to how one can design meaningful interactive experiences in a museum space that addresses sustainability.
}

In the next section I will go through and describe the "steps" (or "moving") I have taken during the thesis project, and how theoretical practise is weighted against (actual) experiences and critical analysis of a set of interactive exhibitions and installations.


\section{How the thesis have developed over time}

I this section I want to give reason as to why and how the thesis research question have changed 




To begin with, the

\section{Model of interaction design research}
Throughout this thesis project, there has been a major directional shift that has affected the research outcome - something which happened right about in the middle of things. I started this thesis project with a pretty clear goal to prototype and design an installation, but have ended up with a critical study of a number of interactive exhibitions and installations that I analyse to identify meaningful relationships or qualities in a museum space. To better show and talk-through this shift, I will use Fallman's model of interaction design research, which I often refer to as 'the design triangle', to better describe how my researching lens have shifted throughout and during the thesis project. It is also a means to explain as to why and how this thesis is fitted and give back to the interaction design research field.

As a design discipline, interaction design’s core can be found in an orientation towards the shaping of digital artefact, products, services, and spaces - with particular attention paid to the qualities of the user experience \autocite[p. 4]{fallman_triangle_2008}. In Fallman’s use of the model, the most interesting and rewarding results in interaction design research come not from taking a specific position in the model, but rather from moving or drifting in between different positions. Thus, as Fallman describe it, "moving in between different positions in the model is, more than anything else, a change of perspective" \autocite[p. 10]{fallman_triangle_2008}.


\begin{figure}[H]
\includegraphics[width=13cm]{pictures/process/triangle.png}
\caption{"The model of interaction design research in its most basic form."}
\autocite[p. 5]{fallman_triangle_2008}
\centering 
\end{figure}

% In terms of doing research through design as a method for interaction design, Zimmerman et. al, explains that what is unique to research though design as an approach, is that it sees the design artefacts as outcomes that can transform the world from its current state to preferred state, which aligns with the domain problems I have accounted for in section 2.0, to address sustainability issues for a more sustainable future. Zimmermann et. al. further explain how the artefacts produced in this type of research become design examples, providing an appropriate channel for research findings to easily transfer to the HCI research and practise communities(Zimmermann et. al, p. 1, 2007).


Design study entails making space for reflections in some kind of structured way on one’s activities: organising reading circles and seminars; and opening up arenas for theoretical, methodological, and philosophical discussions to take place \autocite[p. 18]{fallman_triangle_2008}. The way I have gone forward with this, is to read up on museum practise as evident in chapter 1: Museums, as well as on the topic of sustainability, linking the museum practise up against sustainability. Specifically looking at how sustainability represents a contemporary discourse, and discussing this in relation to how museums want to address and disseminate more contemporary issues to stay relevant. 

+ Design exploration
+ Design Practise


\section{Qualitative research (very unfinished!)}
jeg gjør kritisk forskning: critical design of existing artifacts instead of designing my own. To proper explore Rq I would need t omake a lot of designs, not enough time, better to lean on theory and explore existing installations so that I can actually explore more of them and "talk back to theory". I use theory to analyse installations installations with a design perspective: user experience (meaningful) and dialogical interactive elements/ qualities. 

The paper offers theoretical support for research through design (RtD) by arguing that to legitimise and make use of research through design as \emph{research}, HCI researchers need to explore and clarify how RtD objects contribute to knowledge \autocite[p.2093]{bardzell_immodest_2015}.


Along these lines, Bardzell et.al argue that while the \emph{intentions} of the object's designer are important and annotations are a good mechanism to articulate them, the critical reception of objects can be equally generative of RtD's knowledge impacts \autocite[p. 2093]{bardzell_immodest_2015}.

Bardzell et.al. investigate RtD in its relation to the production of knowledge; specifically, \emph{how design objects are knowledge producers both for those that encounter them and those that design them} \autocite[p. 2093]{bardzell_immodest_2015}. 

To explore how a detailed critique might work when we understand objects as knowledge producers offering to the viewer the possibility to engage in meaning-making practises unfolding a range of complex and multi-faceted views, Bardzell et. al offer a multilevel analysis of a critical design fiction. \autocite[p. 2094]{bardzell_immodest_2015}.

As a field, HCI must answer what sorts of knowledge outcomes can come from objects in (art and) design projects; if we cant, we cannot legitimize RtD as a way of doing HCI research.

My knowledge outcomes from this thesis project:
\begin{itemize}
    \item Proposal of a theoretical lens to read and understand a interactive installations in the museum/ an exhibition. To identify meaningful relations between the dissemination and exhibition practise.
    \item A critical view on the topic of interactivity addressing contemporary topics/issues in a museum space
    \item 
\end{itemize}

This research practise in this thesis is built upon qualitative data through qualitative methods. 

"ethnographic methods" in this thesis: observation, photographic work, interview, conversations, and thinking. 

this thesis has by no means been structured in a read-collect-write linear type of structure.. 

detached researcher? 
What have I been looking for?

Pure subject, what/ who is my subject-object of study?