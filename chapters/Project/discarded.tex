\section{Qualitative research (very unfinished!)}
Flytte til diskusjon? Heller skrive ut om literature review steget?
jeg gjør kritisk forskning: critical design of existing artifacts instead of designing my own. To proper explore Rq I would need t omake a lot of designs, not enough time, better to lean on theory and explore existing installations so that I can actually explore more of them and "talk back to theory". I use theory to analyse installations installations with a design perspective: user experience (meaningful) and dialogical interactive elements/ qualities. 

The paper offers theoretical support for research through design (RtD) by arguing that to legitimise and make use of research through design as \emph{research}, HCI researchers need to explore and clarify how RtD objects contribute to knowledge \autocite[p.2093]{bardzell_immodest_2015}.


Along these lines, Bardzell et.al argue that while the \emph{intentions} of the object's designer are important and annotations are a good mechanism to articulate them, the critical reception of objects can be equally generative of RtD's knowledge impacts \autocite[p. 2093]{bardzell_immodest_2015}.

Bardzell et.al. investigate RtD in its relation to the production of knowledge; specifically, \emph{how design objects are knowledge producers both for those that encounter them and those that design them} \autocite[p. 2093]{bardzell_immodest_2015}. 

To explore how a detailed critique might work when we understand objects as knowledge producers offering to the viewer the possibility to engage in meaning-making practises unfolding a range of complex and multi-faceted views, Bardzell et. al offer a multilevel analysis of a critical design fiction. \autocite[p. 2094]{bardzell_immodest_2015}.

As a field, HCI must answer what sorts of knowledge outcomes can come from objects in (art and) design projects; if we cant, we cannot legitimize RtD as a way of doing HCI research.

My knowledge outcomes from this thesis project:
\begin{itemize}
    \item Proposal of a theoretical lens to read and understand a interactive installations in the museum/ an exhibition. To identify meaningful relations between the dissemination and exhibition practise.
    \item A critical view on the topic of interactivity addressing contemporary topics/issues in a museum space
    \item 
\end{itemize}

This research practise in this thesis is built upon qualitative data through qualitative methods. 

"ethnographic methods" in this thesis: observation, photographic work, interview, conversations, and thinking. 

this thesis has by no means been structured in a read-collect-write linear type of structure.. 

detached researcher? 
What have I been looking for?

Pure subject, what/ who is my subject-object of study?