\section{Qualitative research (very unfinished!)}
Flytte til diskusjon? Heller skrive ut om literature review steget?
jeg gjør kritisk forskning: critical design of existing artifacts instead of designing my own. To proper explore Rq I would need t omake a lot of designs, not enough time, better to lean on theory and explore existing installations so that I can actually explore more of them and "talk back to theory". I use theory to analyse installations installations with a design perspective: user experience (meaningful) and dialogical interactive elements/ qualities. 

The paper offers theoretical support for research through design (RtD) by arguing that to legitimise and make use of research through design as \emph{research}, HCI researchers need to explore and clarify how RtD objects contribute to knowledge \autocite[p.2093]{bardzell_immodest_2015}.


Along these lines, Bardzell et.al argue that while the \emph{intentions} of the object's designer are important and annotations are a good mechanism to articulate them, the critical reception of objects can be equally generative of RtD's knowledge impacts \autocite[p. 2093]{bardzell_immodest_2015}.

Bardzell et.al. investigate RtD in its relation to the production of knowledge; specifically, \emph{how design objects are knowledge producers both for those that encounter them and those that design them} \autocite[p. 2093]{bardzell_immodest_2015}. 

To explore how a detailed critique might work when we understand objects as knowledge producers offering to the viewer the possibility to engage in meaning-making practises unfolding a range of complex and multi-faceted views, Bardzell et. al offer a multilevel analysis of a critical design fiction. \autocite[p. 2094]{bardzell_immodest_2015}.

As a field, HCI must answer what sorts of knowledge outcomes can come from objects in (art and) design projects; if we cant, we cannot legitimize RtD as a way of doing HCI research.

My knowledge outcomes from this thesis project:
\begin{itemize}
    \item Proposal of a theoretical lens to read and understand a interactive installations in the museum/ an exhibition. To identify meaningful relations between the dissemination and exhibition practise.
    \item A critical view on the topic of interactivity addressing contemporary topics/issues in a museum space
    \item 
\end{itemize}

This research practise in this thesis is built upon qualitative data through qualitative methods. 

"ethnographic methods" in this thesis: observation, photographic work, interview, conversations, and thinking. 

this thesis has by no means been structured in a read-collect-write linear type of structure.. 

detached researcher? 
What have I been looking for?

Pure subject, what/ who is my subject-object of study?

\section{Interview with a concept developer from Munch}
\par
\emph{date date date}
\par


\emph{"Det beste er jo når folk lærer noe nytt uten at de merker det, det burde jo være målet og det kan jo være tekst for eksempel som byr på en veldig engasjerende historie men det kan også være noe interaktivt, f.eks. du tar en selfie og at Munch også gjorde det og lagde disse effektene, og kanskje man blir litt interessert i det og kan finne ut mer om det andre steder i museet, som f.eks. i “uendelig utstilling” er det også sånn at vi mener at de forskjellige utstillinger er jo ..(?) av hverandre (at de bygger på hverandre?) og skaper altså verdi for hverandre, at folk som har vært i skygger har muligheten å oppleve Munchs kunst på nye måter ved gå inn i uendelige [avdeling]. Men vi jobber også med nye utstillingsprosjekter der vi, der den tilknytningen mellom digitale opplevelser og mer tradisjonelle presentasjoner av kunst er mer tettere, f.eks. at du at man har sånne immersive rom eller opplevelser i en utstillingssal man kan gå inn i og så ut igjen og oppleve kunsten på en ny måte. Det syns vi er en veldig produktiv måte for å åpne opp utstillingssaler for et større publikum og engasjere flere og bredere. [Her sier B: ja flere perspektiver eller, men tok ikke det på egen linje]. Ja. Vi ser det også med Poison [utstilling] f.eks., mange som likte det hadde også behov for å se originalbildene f.eks. Så det tror jeg er absolutt noe som ikke erstatter originalkunsten, som mange frykter, men må være som beriker og gir deg kanskje en, gir også visse grupper en sjans for at de gidder å befatte seg med det eller se på kunsten og ikke bare, de fleste ser jo bare i 3 eller 5 sekunder på et bilde og så gå videre til neste. Uten formiddling er det ikke noen sjanse for å øke den tiden enkelte tilbringer med kunsten."}


\section{from klimahuset bservation}
Med en klimavert til stede får man bedre hjelp og en form for retning til å “lese” installasjonene, som støtter deg når du senere går gjennom avdeling for avdeling. 
Man blir stilt spørsmål som er knyttet til faktiske ting, som for eksempel i video-atriet sier klimavert; jeg er 160cm høy, hvor høy tror dere denne veggen er?
Etter litt håndsopprekning får vi vite at dersom Grønlandsisen smelter vil havet stige like høyt som det den veggen er. Det er tankevekkende og setter inntrykk!

Her blir man også litt senere spurt: ser dere vær eller klima? et åpent spørsmål som setter i gang en god diskusjon på forskjellen mellom vær og klima, med fokus på hvordan vær påvirker klima. Dette blir forklart at mange blander og tenker at vær og klima er det samme, og at klimaforandringer og værforandringer ikke er det samme. Her får noen elever aha-øyeblikk.


Et annet eksempel er isbiten inne i “den naturlige avdelingen”.  Først får elevene i oppgave å finne noe i området som påvirker klimaet naturlig. Senere når det blir tatt en liten runde spør verten; har alle tatt på isbiten? For så å gå videre til å forklare hvordan menneskelig påvirkning påvirker klimaet. “For eksempel har deres varme hender bidratt til å smelte litt av isbiten her i Klimahuset”. Dette er også tankevekkende og setter inntrykk!