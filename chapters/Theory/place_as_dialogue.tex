\section{A dialogical approach to place}
John McCarthy and Luigina Ciolfi presents what they call a dialogical approach to place, people and technology. Dialogical is an adjective relating to or in the form of dialogue. I found this relevant in light of my earlier inquiry into the climate debate (the public discussion) on how to solve climate- related issues, to the articles focus on the technological mediation of museum as a place in which people encounter and experience exhibited artefacts curated to fit the museums expository agency.

> Experience involves acting and being acted upon, sensing and feeling both, and transforming them into something emotionally and intellectually meaningful. The sense people make of their experience—individual and collective—makes use of place and indeed makes or transforms place. As sensory and affective experience becomes transformed in thought and story, a museum (or anyother environment) can become a significant place for people and contributes in some meaningful way to transforming the people themselves \autocite[p. 250]{mccarthy_place}.

In Ciolfi and McCarthy’s opinion, frameworks that have been made to guide the design of interactive museum exhibitions developed in the field of museums studies, underplay aspects of visitor’s active sense making and interpretation. And that most practical and conceptual contributions from both museum studies and interaction design have fallen short of their potential to reflect on and design technologically mediated museum experiences partly because of the underdeveloped or under-articulated conceptualisations of visitor experience with which they work (McCarthy and Ciolfi, 2008, p. 248). 




\section{Five dialogic principles}
The dialogical approach attends to the complexity and plurality of experience of place both as material and ideal, physical and cultural, sensory and reflective. Thus it suggests that analysis of experience of a place requires attention to both the immediate sensory transaction and the ways in which the immediate experience transforms in the telling (McCarthy and Ciolfi, 2008, p. 251). They propose the following dimensions as building blocks of what they call a dialogical ontology that can prove to be useful in the design of and evaluation of interactive museum experiences:



\emph{Museum Experience is Relational.} Experience is seen in terms of the variety of relationships and practices of which it is constituted. One way to look at this variety is in terms of the relationships that sustain different museums, for example relationships between past and present, the building and the community, museum staff and visitors, exhibits and visitors. 
\par
\emph{Museum Experience is Open.} There is a sense in which the very idea of a digital artefact or installation in a museum plays on the boundary between two contrasting genres; fx digital and traditional. Beyond these and more genres, some museums play with the openness of experience by creating areas for enquiry, study, questioning and discussion, places that actively promote dialogue.
\par
\emph{Experiences in a Museum are at the Centre of a Variety of Sense-making Practises.} Environments, objects, artefacts and indeed exhibitions and museums attain meaning for people through their ongoing experience with and reflection on them. We interpret the situation in terms of our previous experiences and we reflect on our experience and our response to it. These processes give our experiences a narrative quality.
\par
\emph{Museum Experience Situates Artefacts in Narrative.} The narrative around them can be as important to experiencing them as the objects themselves. There is a sense in which we actually ‘make’ the experience by recounting it, as the expression of experience is both structured by and structures the experience. 
\par
\emph{Museum Experience is Sensitive to the Peculiarities of Space and Time.} Attending to the ways in which we make sense of experience introduces temporal and social dimensions to our account of experience. The temporal refers to the ways in which past and future are folded into the present experience as we make sense of it. Becoming present also changes past and future experience. We have seen how anticipating a future that includes explaining current experience to others changes current experience. Of course, it also transforms what future experience might be as we fill the future with those imagined explanations and encounters, which in time becomes the past transforming the present. 

I agree with and like where Ciolfi’s research is going, and have found Ciolfi and McCarthy’s dialogical dimensions useful as a supplement as to where I can include this as a perspective of the visitors museum experience, in the ongoing analysis of the four museum dimensions presented in section 3.2 designing hybrid places. I also find that these dialogical dimensions can prove useful when designing and conceptualize the interactive installations for this thesis. I believe these dialogical dimensions can give me the extended vocabulary to further explore and define specific concepts linked to the problem framings from section 1.1:

How can exhibition artefacts invite visitors to come together and discuss sustainability issues?\par
Can interactive installations encourage visitors to take more action in sustainability issues after the exhibition visit?\par
Can new/different interactions with natural objects contribute to increased climate consciousness and activism?\par


\section{Discursive qualities in a museum}
\par
\emph{Or: cultural analysis in museums}
\par

This thesis have been highly influenced by Mieke Bal. Mieke Bal is a dutch cultural theorist, video artist and Professor in Literary Theory at the University of Amsterdam, with academic interest and background in humanities, media and culture studies. From her writings on the discourse of the museum, she discusses what differentiates the “new” museology from the “old”, and presents the museum as a discourse and the exhibition as an utterance within that discourse. Bringing this discursive perspective to the museum deprives the museal practise of its innocence, and provides it with the accountability it and its users are entitled to (Thi, p. 214). Part of her argument is that politics come straight out of, or more precisely are bound up with, the museal discourse (Thi, p. 214), and proposes a threefold direction to museology researchers. First she suggest to systematically analyse the narrative-rhetorical structure of the specific museum, in order to refine the categories and deepen insight into their effects. Secondly she suggest to look at the connection between the museal discourse and the institutions foundation and history, and thirdly she see the need to do self-critical analysis of the museal discourse as a consequence of the nature of discourse.

With my background and scope of thesis it is both irrelevant and I am by no means capable to do a discursive analysis in the way Mieke Bal proposes. I do however find aspects of Bal’s discursive perspective relevant, like her proposal of how one can do a narrative-rhetorical systematic analysis of a specific museum. She provides a set of new terms and vocabulary so that I better can “read”, describe, understand and do research on a specific museum and/ or specific installation. I find this perspective along with the extended vocabulary it provides useful so that I better can identify meaningful relations between user activity at installations/ artefacts and the museum experience.

To help look at what happens in a museum, Bal is guided by the notion that the gesture of showing can be considered a discursive act. According to her, exhibiting a subject through putting “things” on display, creates a subject/object dichotomy (Bal, p. 3). Dichotomy is a division or contrast between two things that are represented as being opposed or entirely different. The dichotomy enables the subject to make a statement about the object, while the object is there to prove evidence or the truth of the statement. The addressee for the statement is the visitor, viewer or reader (Bal, p. 3). The discourse surrounding the exposition, or, more precisely the discourse that is the exposition, is “constative”: informative and affirmative (Bal, p. 3). The very fact of exposing the object —presenting it while informing about it — impels the subject to connect the “present” of the object to the “past” of their making, functioning and meaning (Bal, p. 4). This is one of the levels giving the exposition a narrative quality. The other level where narrative can occur is the necessarily sequential nature of the visit, the “walking tour” (Bal, p.4). The “walking tour” bind the elements of the exposition for the viewer together, where the narrative of walking through a museum is comparable to the narrative of reading a book.

The reason for Bal’s attention to museums is the actual, concrete “literal” exhibition of things in museums and galleries. She proposes that the tools to conduct cultural analysis in museums, are best selected and employed as an integration of rhetoric and narratology. Rhetoric helps to “read” not just the artefacts in a museum, but also the museum and its exhibitions themselves (Bal, p. 7). While the narratological perspective provides meaning to the otherwise loose elements of such a reading (Bal, p. 7). Most importantly, the analysis aims to yield, one the one hand an integrated account of the discursive strategies put into effect by the museums expository agent (the curators), and, on the other hand, the effective process of meaning-making these strategies suggest to the visitor (Bal, p.7). The reading itself, then, becomes part of the meaning it yields.

(For my study of how one can design meaningful, interactive experiences in a museum space that addresses sustainability, this perspective is relevant because it enables me to undersøke the sustainability domain with the help of vocabulary derived from museology and cultural analysis. In similar museums, like heritage museums displaying ethnic types of discourses - cultural analysis has helped identify cultural moralism and imperialism in museums. I have a hypothesis that cultural analysis in museums addressing sustainability issues could reveal power structures or tensions in the museums dissemination and exhibitory practise. Or perhaps moralism?)

The book account for the museum itself being an expository agent and having an expository agency, and the exposition, both in the broader, general sense of “exposing an idea”, and in the specific sense of being an exhibition pointing at objects. By dissecting and putting forward the expository agent who is implicitly “speaking” throughout the exhibition design, about the objects on display, one can see how one comments on the other. The book also explains how different museums speak different “fictions” as they call it; where fx the Munch museum display art and is namely an art museum. Klimahuset display climate history and futuristic scenarios, and as I have read the book would classify it as an ethnographic museum. The genre to which a museum allegedly belongs to is in itself part of the frame that puts pressure on the meanings it will produce. The distinction between the ethnographic museum and the art museums can help us understand the importance of addressing this central discursivity and of interpreting its effects. The ethnographic museum conserves and exhibits artefacts, the art museum, works of art.

\break
From Mieke Bal’s book: Double exposures - the subject of cultural analysis:
\begin{itemize}
    \item Who is speaking? expository agents & expository agency
    \item What is spoken? telling, showing or showing off?
    \item Museology vs museums and museum fictions
    \item Difficulties of looking and the need to read: artwork vs artefact vs object on display
\end{itemize}


These (over) are the terms I take with me into the framework!
