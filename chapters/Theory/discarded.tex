The use of theory is essential to any academic discipline. Before we embark on to the Theory chapters, it is necessary to go into how theory have been put to use throughout the thesis project. Theory takes researchers beyond observation and interpretation into the realm of shareable knowledge \autocite[p. 126]{beck_examining_2016}. It provides us with the means to structure knowledge, to evaluate and assess it, to construct it, and to share it \autocite[p. 126]{beck_examining_2016}. In a paper examining the practical, everyday theory use in design research, Beck and Stolterman discuss how researchers put theories to work in their written texts, and synthesise six models of "theory use". The models reflect the different ways researchers make use of theory beyond the commonly referenced uses of explanation, generalisation, prediction, and the like. And shows how theory can be used to motivate inquiry, contextualize research, shape research questions and guide methodology and analysis \autocite[p. 134]{beck_examining_2016}. The six models they propose is namely: no theory, theory as the object of study, theory as a contextualising tool, theory as a shaping tool, theory as a methodological tool, and theory as an analytical tool. Theory has primarily been put to use as a contextual and analytical tool in this thesis. 


\par \emph{Theory as an object} \par
"When a researcher develops an explanation of how or why some phenomenon occurs, they are engaged in theorizing a \emph{process}" \autocite[p.126]{beck_examining_2016}. The explanation itself becomes a theoretical \emph{object} \autocite[p. 126]{beck_examining_2016}.  


\par \emph{Theory as a tool} \par
Framing theory as a tool implies that a user uses theory for a particular purpose. Theory has been described and defined by many researchers in terms of its utility. It has been framed as a tool for explaining, describing, or predicting phenomena. It has been described in design research as a tool for "binding together" our knowledge of design practise and as a tool for "providing an understanding" of design writ large. \autocite[p. 127]{beck_examining_2016}. Similarly, theory is not necessarily an ideal tool for explaining reality in the same way that an analogy or metaphor might be. THeory, a tool for explaining, predicting, or describing, has structural properties conducive to other purposes as well. \autocite[p. 127]{beck_examining_2016}.


\par \emph{Theory as a reference} \par
Common definitions of reference include: the action of mentioning or alluding to something or the use of information to ascertain something. When theory is used as a reference, it most often appears in the introductory and/or background section of a given wrritten text \autocite[p. 128]{beck_examining_2016}. With this type of of theory use, researchers often reference frameworks and models \emph{instead} of referencing theory per se. Theory as a reference can perform several functions that may appear in concert or individually in a given research paper. For instance, a theory can be used to etablish the basis for a research project, to situate a text within a lineage, to establish the knowledge base of the writer, or to show connection to a community or school of thought. 
\par

(Mieke Bal narrative theories etc is an example of how  have used theory as a reference!)

\par \emph{Theory as a knowledge contribution} \par
Characterising theory as a \emph{knowledge contribution} suggests that it had not existed prior to the researcher's articulation of it. The researchers brought it into being by formulating their results such that they would be understood as a theory, for instance, as a set of constructs and their definitions, as well as a set of propositions about how the construct relate to one another. This formulation of a theory becomes a knowledge object. \autocite[p. 128]{beck_examining_2016}.




\section{Annotated portfolio}
% New title in section: A framework to analyse annotated portfolios?

"Designed artefacts lie at the intersection of multiple considerations. Some of these concern the functionality of the design (what should it do?), some the aesthetics (what form and appearance should the artefact take?), some concern the practicalities of its production (what materials, skills and tools are needed to make it?), others concern the motivation for making (why are we doing this? what are we trying to show). Yet others concern the identities and capabilities of the people for whom the artefact is intended (what will our users make of this? how can we best design for them?)" \autocite[p. 70]{bowers_annotated_2012}. 

'Annotated portfolio' is defined in terms of seven features (constitution, relationships, communication, perspective, mutual informing, shaping and materiality) \autocite[p. 71]{bowers_annotated_2012}. "As long as the principles just articulated (...) are attended to, any material form can be considered" \autocite[p. 73]{bowers_annotated_2012}.

"Annotations make a collection into a portfolio. A collection might be from a single designer, project or studio, though it need not be. Portfolios are typically more than just a collection of works. Works are organised, categorised and otherwise arranged in their presentation to have or illustrate a point, or several points, and through doing this to reveal or make clear something of the design identities in the work and the nature of the contribution being made." \autocite[p. 71]{bowers_annotated_2012}. "Annotations are a major resource for creating a portfolio. Works do not speak for themselves. They are annotated so as to show how they fit into a portfolio of related endeavour" \autocite[p. 71]{bowers_annotated_2012}.

"Annotations capture family resemblances between designs. In a portfolio, different works have a resemblance to one another but it need not be the case that one single feature has to run through all works in a portfolio." \autocite[p. 71]{bowers_annotated_2012}. "An annotated portfolio is a pragmatic thing. If not an abstractly organised collection of work. Bower say that; "how we annotate and how we select works for inclusion in a portfolio reflects interests and purposes". Interests and purposes are future-looking. \autocite[p. 73]{bowers_annotated_2012}.

"Annotated portfolios are proposed as a major way in which RtD might cash out its value for the communities we typically engage with (users, designers, scientists, and others of course). Annotated portfolios do, on this view, much of the work traditionally expected of 'Theory' " \autocite[p. 73]{bowers_annotated_2012}.
