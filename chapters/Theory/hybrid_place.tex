\section{Designing hybrid places}

Ciolfi & Bannon (2007) discusses and presents how a place-centred approach can be utilised in a practical manner to inform the design of interactive environments. They have shown how new information technologies, when used in museum environments, often suffer from a number of drawbacks; in particular, they argue that these technologies may hinder visitors appreciation of museum artefacts, their social interaction with others, and their appreciation of the place (Ciolfi & Bannon, 2007, p. 178). Research focusing on the design of novel interactive artefacts has recently become more concerned with studying and understanding the spatial properties of the world where it is placed (Ciolfi & Bannon, 2007, p. 159), aimed at understanding and encompassing more of the physical environment where interactions occur.

I find this place-centred approach helpful in terms of answering how one can design interactive, meaningful experiences, because it help position and argue for why and how the context (surroundings, environment) play an important role in strengthening the overall impression of the installation and message conveyed. They argue that a place-centred approach can practically guide and support design so that one can design to extend and support the museum experience. 

My argument/ hypothesis is that it is crucial to take the context the installation is placed in, into account and simultaneously think about the overall visitor experience if you want to have any chance in designing for a meaningful interactive experience. In my opinion, interactivity alone wont build a meaningful experience, it is a close "samspill" between the installations in the narrative path that the museum build, as well as a way to remember the place where you experienced something meaningful. I believe that over time, when perhaps months or years have passed and the details of the experience is forgotten, what remains is the memory of something meaningful happening in that place; the museum. That's what makes you want to come back to the space, or bring friends and family.


	(interactive artefact vs interactive experience)
Research focusing on the design of novel interactive artefacts has recently become more concerned with studying and understanding the spatial properties of the world (Ciolfi & Bannon, 2007, p. 159). Bringing technologies beyond the desktop and into the world requires an ever-increasing interest in the physical environment where interaction occurs (Ciolfi & Bannon, 2007, p. 159), 

	We are located in ‘space’, but we act in ‘place’
The article gives a thorough discussion and (to me, introduction) on viewpoints on the difference between place and space. Ciolfi & Bannon, 2007, p. 161;



\emp{Place is a space which is invested with understanding of behavioural appropriateness, cultural expectations and so forth. We are located in ‘space’, but we act in ‘place’. }(Harrison and Dourish, 1996)

For example, Munro, Höök and Benyon(1999), discuss several conceptions of place noting how the concept of place in architecture can inform the design of information spaces (Ciolfi & Bannon, 2005, p. 221). 

In the paper, Ciolfi & Bannon applies what they call a geographical notion of place to the study of a particular physical environment (a museum), for the purpose of designing an interactive museum exhibition (Ciolfi & Bannon, 2007, p. 161-162). And, as to whereas space refers to dimensions involving abstract geometrical extension and location, place describes our experience of being in the world and investing a physical location or setting with meaning, memories and feelings (Ciolfi & Bannon, 2007, p. 162). Following these definitions which they derived from Yi-Fu Tuan’s Geography, phenomenology, and the study of human nature, 1971. (Our consideration of the multiple dimensions of people’s experience, allowed us to pinpoint issues to be dealt with in the scenario design phase. (Ciolfi & Bannon, 2007, p. ??))

 Ciolfi & Bannon articulate four dimensions of place:

BRUK UTFYLTE PATTERNS FRA SIDE 173!!!
\par
\emph{The physical dimension.} Relating to materials, structures and environmental factors.
\par
\emph{The personal dimension.} Related to the feeling and emotions we associate to a place, to the memories related to or evoked by it, to the personal knowledge and background we invest the place with while making sense of it.
\par
\emph{The social dimension.} Related to social interaction and communication within the place, to the sharing of resources and memories, to social co-ordination and ethics, etc.
\par
\emph{The cultural dimension.} Related to the rules, conventions and cultural identity of a place and of its inhabitants.
\par

I intend to use these four dimensions as a foundation for my scope in terms of data-gathering in the different museums and exhibitions that I’m visiting this semester. Ciolfi & Bannon explains how each dimension is present at any moment of one’s experience of a place, and the experience is shaped by the dynamic interconnections among these dimensions(p. 162), and after my museum visits I want to be able to analyse the museums and be enabled to identify some interconnections between the physical, personal, social and cultural dimension in a museum.

I also intend to use the four dimensions in the synthesising of the framework of how to design meaningful interactive museum experiences. 


page 168 in hybrid place article they articulate four pitfalls or "problems" that arise:
- How people appreciate the exhibit
- how people interact with eachother; and
- how people devise their own path through the exhibits and leave a trace of their presence in the space.

\section{Place as dialogue}

John McCarthy & Luigina Ciolfi presents what they call a dialogical approach to place, people and technology. Dialogical is an adjective relating to or in the form of dialogue. I found this relevant in light of my earlier inquiry into the climate debate (the public discussion) on how to solve climate- related issues, to the articles focus on the technological mediation of museum as a place in which people encounter and experience exhibited artefacts curated to fit the museums expository agency.

> Experience involves acting and being acted upon, sensing and feeling both, and transforming them into something emotionally and intellectually meaningful. The sense people make of their experience—individual and collective—makes use of place and indeed makes or transforms place. As sensory and affective experience becomes transformed in thought and story, a museum (or anyother environment) can become a significant place for people and contributes in some meaningful way to transforming the people themselves. (McCarthy & Ciolfi, 2008, p. 250).

In Ciolfi & McCarthy’s opinion, frameworks that have been made to guide the design of interactive museum exhibitions developed in the field of museums studies, underplay aspects of visitor’s active sense making and interpretation. And that most practical and conceptual contributions from both museum studies and interaction design have fallen short of their potential to reflect on and design technologically mediated museum experiences partly because of the underdeveloped or under-articulated conceptualisations of visitor experience with which they work (McCarthy & Ciolfi, 2008, p. 248). 

The dialogical approach attends to the complexity and plurality of experience of place both as material and ideal, physical and cultural, sensory and reflective. Thus it suggests that analysis of experience of a place requires attention to both the immediate sensory transaction and the ways in which the immediate experience transforms in the telling (McCarthy & Ciolfi, 2008, p. 251). They propose the following dimensions as building blocks of what they call a dialogical ontology that can prove to be useful in the design of and evaluation of interactive museum experiences:

\emph{Museum Experience is Relational.} Experience is seen in terms of the variety of relationships and practices of which it is constituted. One way to look at this variety is in terms of the relationships that sustain different museums, for example relationships between past and present, the building and the community, museum staff and visitors, exhibits and visitors. 
\par
\emph{Museum Experience is Open.} There is a sense in which the very idea of a digital artefact or installation in a museum plays on the boundary between two contrasting genres; fx digital and traditional. Beyond these and more genres, some museums play with the openness of experience by creating areas for enquiry, study, questioning and discussion, places that actively promote dialogue.
\par
\emph{Experiences in a Museum are at the Centre of a Variety of Sense-making Practises.} Environments, objects, artefacts and indeed exhibitions and museums attain meaning for people through their ongoing experience with and reflection on them. We interpret the situation in terms of our previous experiences and we reflect on our experience and our response to it. These processes give our experiences a narrative quality.
\par
\emph{Museum Experience Situates Artefacts in Narrative.} The narrative around them can be as important to experiencing them as the objects themselves. There is a sense in which we actually ‘make’ the experience by recounting it, as the expression of experience is both structured by and structures the experience. 
\par
\emph{Museum Experience is Sensitive to the Peculiarities of Space and Time.} Attending to the ways in which we make sense of experience introduces temporal and social dimensions to our account of experience. The temporal refers to the ways in which past and future are folded into the present experience as we make sense of it. Becoming present also changes past and future experience. We have seen how anticipating a future that includes explaining current experience to others changes current experience. Of course, it also transforms what future experience might be as we fill the future with those imagined explanations and encounters, which in time becomes the past transforming the present. 

I agree with and like where Ciolfi’s research is going, and have found Ciolfi & McCarthy’s dialogical dimensions useful as a supplement as to where I can include this as a perspective of the visitors museum experience, in the ongoing analysis of the four museum dimensions presented in section 3.2 designing hybrid places. I also find that these dialogical dimensions can prove useful when designing and conceptualize the interactive installations for this thesis. I believe these dialogical dimensions can give me the extended vocabulary to further explore and define specific concepts linked to the problem framings from section 1.1:

How can exhibition artefacts invite visitors to come together and discuss sustainability issues?\par
Can interactive installations encourage visitors to take more action in sustainability issues after the exhibition visit?\par
Can new/different interactions with natural objects contribute to increased climate consciousness and activism?\par

\section{Technologically enhanced physical environments}
The perspective of “Ubiquitous Computing”, proposed in the early 1990’s by Mark Weiser, is based on technological developments that make it possible to embed powerful computational elements and digital components into everyday objects, portable devices and the built environment. This trend is inducing significant changes not only in the development and implementation of new technology, but also, and more interestingly, on the relationship between interactive systems and their users. (Ciolfi & Bannon, p. 217). Design must now concern itself rather with the physical environments that people will experience through their daily lives. People will encounter technologically enhanced spaces and artefacts as they move through a variety of environments. (p. 217). These systems will change the way in which physical spaces are used and shaped by people, where the systems are able to react and respond to their presence and actions. The activities of interacting with the space and its elements and interacting with the computer system will merge into each other. (Ciolfi & Bannon, 2005, p. 217). 

The Interaction Design field is currently ongoing a shift in the understanding of the relationship between people and technologies, where HCI primarily have been focused on a single users individual traits and preferences to computer functionalities and interface elements, a one-to-one relationship with the computer system. Interaction Design is also corned with the role of social, emotional and contextual factors in influencing human interaction with a computer system. (Ciolfi & Bannon, 2005, p. 217).

Context-aware systems are those that are able to sense features of the physical setting, and to feed a representation of this data into the system itself. The sensing devices can be located both around the physical environment and on the bodies of the inhabitants( Ciolfi & Bannon, 2005, p. 218.) An example of an enhanced space is that of “Narrative Spaces” (Sparacino 2002), where users are able to produce “augmented” performances on an interactive stage with the support of sensors and gesture modeling tools: the performers physical movements trigger the production of sounds and computer-generated images on large display screens. (Ciolfo & Bannon, 2005, p. 219). 

Ciolfi & Bannons research aim to apply the concept of place to a particular set of ubiquitous systems: technologically-enhanced physical spaces. With ubiquitous technologies becoming more reliable and widespread, we are now dealing with fully interactive physical spaces, containing tangible elements acting as interfaces to access features of the digital domain. We believe, as notes by Ciolfi & Bannon, the concept of place can assist interaction designers to understand interaction dynamics in this context, and to propose effective design concepts: where place goes beyond the vision of space just as a physical setting, a container, and includes many dimensions of human experience within an environment. (Ciolfi & Bannon, 2005, p. 221).
