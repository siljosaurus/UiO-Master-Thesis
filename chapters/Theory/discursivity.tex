\section{Discursivity}
\par
\emph{Or: cultural analysis in museums}
\par

This thesis have been highly influenced by Mieke Bal. Mieke Bal is a dutch cultural theorist, video artist and Professor in Literary Theory at the University of Amsterdam, with academic interest and background in humanities, media and culture studies. From her writings on the discourse of the museum, she discusses what differentiates the “new” museology from the “old”, and presents the museum as a discourse and the exhibition as an utterance within that discourse. Bringing this discursive perspective to the museum deprives the museal practise of its innocence, and provides it with the accountability it and its users are entitled to (Thi, p. 214). Part of her argument is that politics come straight out of, or more precisely are bound up with, the museal discourse (Thi, p. 214), and proposes a threefold direction to museology researchers. First she suggest to systematically analyse the narrative-rhetorical structure of the specific museum, in order to refine the categories and deepen insight into their effects. Secondly she suggest to look at the connection between the museal discourse and the institutions foundation and history, and thirdly she see the need to do self-critical analysis of the museal discourse as a consequence of the nature of discourse.

With my background and scope of thesis it is both irrelevant and I am by no means capable to do a discursive analysis in the way Mieke Bal proposes. I do however find aspects of Bal’s discursive perspective relevant, like her proposal of how one can do a narrative-rhetorical systematic analysis of a specific museum. She provides a set of new terms and vocabulary so that I better can “read”, describe, understand and do research on a specific museum and/ or specific installation. I find this perspective along with the extended vocabulary it provides useful so that I better can identify meaningful relations between user activity at installations/ artefacts and the museum experience.


\section{Vocabulary}
To help look at what happens in a museum, Bal is guided by the notion that the gesture of showing can be considered a discursive act. According to her, exhibiting a subject through putting “things” on display, creates a subject/object dichotomy (Bal, p. 3). Dichotomy is a division or contrast between two things that are represented as being opposed or entirely different. The dichotomy enables the subject to make a statement about the object, while the object is there to prove evidence or the truth of the statement. The addressee for the statement is the visitor, viewer or reader (Bal, p. 3). The discourse surrounding the exposition, or, more precisely the discourse that is the exposition, is “constative”: informative and affirmative (Bal, p. 3). The very fact of exposing the object —presenting it while informing about it — impels the subject to connect the “present” of the object to the “past” of their making, functioning and meaning (Bal, p. 4). This is one of the levels giving the exposition a narrative quality. The other level where narrative can occur is the necessarily sequential nature of the visit, the “walking tour” (Bal, p.4). The “walking tour” bind the elements of the exposition for the viewer together, where the narrative of walking through a museum is comparable to the narrative of reading a book.

The reason for Bal’s attention to museums is the actual, concrete “literal” exhibition of things in museums and galleries. She proposes that the tools to conduct cultural analysis in museums, are best selected and employed as an integration of rhetoric and narratology. Rhetoric helps to “read” not just the artefacts in a museum, but also the museum and its exhibitions themselves (Bal, p. 7). While the narratological perspective provides meaning to the otherwise loose elements of such a reading (Bal, p. 7). Most importantly, the analysis aims to yield, one the one hand an integrated account of the discursive strategies put into effect by the museums expository agent (the curators), and, on the other hand, the effective process of meaning-making these strategies suggest to the visitor (Bal, p.7). The reading itself, then, becomes part of the meaning it yields.

(For my study of how one can design meaningful, interactive experiences in a museum space that addresses sustainability, this perspective is relevant because it enables me to undersøke the sustainability domain with the help of vocabulary derived from museology and cultural analysis. In similar museums, like heritage museums displaying ethnic types of discourses - cultural analysis has helped identify cultural moralism and imperialism in museums. I have a hypothesis that cultural analysis in museums addressing sustainability issues could reveal power structures or tensions in the museums dissemination and exhibitory practise. Or perhaps moralism?)

The book account for the museum itself being an expository agent and having an expository agency, and the exposition, both in the broader, general sense of “exposing an idea”, and in the specific sense of being an exhibition pointing at objects. By dissecting and putting forward the expository agent who is implicitly “speaking” throughout the exhibition design, about the objects on display, one can see how one comments on the other. The book also explains how different museums speak different “fictions” as they call it; where fx the Munch museum display art and is namely an art museum. Klimahuset display climate history and futuristic scenarios, and as I have read the book would classify it as an ethnographic museum. The genre to which a museum allegedly belongs to is in itself part of the frame that puts pressure on the meanings it will produce. The distinction between the ethnographic museum and the art museums can help us understand the importance of addressing this central discursivity and of interpreting its effects. The ethnographic museum conserves and exhibits artefacts, the art museum, works of art.

From Mieke Bal’s book: Double exposures - the subject of cultural analysis:

Who is speaking? expository agents & expository agency
What is spoken? telling, showing or showing off?
Museology vs museums and museum fictions
Difficulties of looking and the need to read: artwork vs artefact vs object on display

These (over) are the terms I take with me into the framework!
